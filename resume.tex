%%%%%%%%%%%%%%%%%%%%%%%%%%%%%%%%%%%%%%%%%
% Medium Length Professional CV
% LaTeX Template
% Version 3.0 (December 17, 2022)
%
% This template originates from:
% https://www.LaTeXTemplates.com
%
% Author:
% Vel (vel@latextemplates.com)
%
% Original author:
% Trey Hunner (http://www.treyhunner.com/)
%
% License:
% CC BY-NC-SA 4.0 (https://creativecommons.org/licenses/by-nc-sa/4.0/)
%
%%%%%%%%%%%%%%%%%%%%%%%%%%%%%%%%%%%%%%%%%

%----------------------------------------------------------------------------------------
%	PACKAGES AND OTHER DOCUMENT CONFIGURATIONS
%----------------------------------------------------------------------------------------

\documentclass[
	%a4paper, % Uncomment for A4 paper size (default is US letter)
	11pt, % Default font size, can use 10pt, 11pt or 12pt
]{resume} % Use the resume class

\usepackage{xcolor}
\definecolor{linkColor}{HTML}{0077B5}
\usepackage{textcomp}
\usepackage{fontspec} % For using system fonts
\setmainfont{Calibri}[
    Path = ./fonts/,
    Extension = .ttf,
    UprightFont = *-regular,
    BoldFont = *-bold,
    ItalicFont = *-italic,
    BoldItalicFont = *-bold-italic
] % Use Calibri font from fonts/ directory
\usepackage[colorlinks=true, urlcolor=linkColor, linkcolor=linkColor]{hyperref} % For clickable links with custom color

%------------------------------------------------

\name{Ben Van Bavel} % Your name to appear at the top

% You can use the \address command up to 3 times for 3 different addresses or pieces of contact information
% Any new lines (\\) you use in the \address commands will be converted to symbols, so each address will appear as a single line.

\address{Researcher \\ Mechanical Engineer \\ Consultant}
\address{\href{https://www.linkedin.com/in/ben-vanbavel/}{linkedin.com/in/ben-vanbavel/}} % Contact information

%----------------------------------------------------------------------------------------

\begin{document}

\begin{rSection}{Personal Profile}
	Self-motivated mechanical research engineer with a strong background in computational mechanics, statistics, and composite materials. 
	Experienced in leading research projects, funding acquisition, collaborating with industry partners, 
	and developing innovative solutions for complex engineering challenges.
\end{rSection}

%----------------------------------------------------------------------------------------
%	WORK EXPERIENCE SECTION
%----------------------------------------------------------------------------------------

\begin{rSection}{Experience}

	\begin{rSubsection}{Research Associate}{March 2025 - Present}{Flanders Make @KU Leuven}{Leuven, BE}
		\item Secured \texteuro350k funding to valorize PhD research into a statistical software tool for industrial use.
		\item Collaborated on a short-term valorization project with industrial partners to develop simulation-based reliability assessment software for mechanical product design.
		\item Managed a \texteuro9.6M European research consortium of 15 partners to develop economical and sustainable composite hydrogen storage vessels, and helped secure its funding.
		\item Supervised 13 MSc theses and 1 early-stage researcher in the field of composite materials, statistical simulation-based design, and reliability assessment.
	\end{rSubsection}

%------------------------------------------------

	\begin{rSubsection}{Doctoral Researcher}{September 2020 - March 2025}{}{}
		\item Led a team of 3 PhD candidates to realize a 4-year VLAIO-funded project named ``OptiVAS'' on reliable composite hydrogen storage vessels.
		\item Collaborated with multinational industry partners (CAE software, automotive OEM, manufacturing) to gather requirements and validate research results.
		\item Developed a novel reliability-based design methodology for composite pressure vessels, reducing material costs by 20\% while maintaining safety standards.
		\item Communicated research findings through 5 journal publications and 7 international conference presentations.
	\end{rSubsection}

\end{rSection}

%----------------------------------------------------------------------------------------
%	EDUCATION SECTION
%----------------------------------------------------------------------------------------

\begin{rSection}{Education}
	
	\textbf{PhD in Mechanical Engineering} \hfill \textit{2020 - 2025} \\ 
	KU Leuven, Belgium \smallskip \\
	Thesis: Reliability-Based Design of Filament Wound Composite Pressure Vessels: Incorporating Multiscale Spatial Material Variability \\

	\textbf{MSc in Mechanical Engineering: Aerospace} \hfill \textit{2015 - 2020} \\
	KU Leuven, Belgium \smallskip \\
	Grade: Magna Cum Laude \\

\end{rSection}

%----------------------------------------------------------------------------------------
%	TECHNICAL STRENGTHS SECTION
%----------------------------------------------------------------------------------------

\begin{rSection}{Competences}

	\begin{tabular}{@{} >{\bfseries}l @{\hspace{6ex}} l @{}}
		Soft Skills & Planning \& Organization, Communication, Leadership \\
		Hard Skills & Statistics, Computational Mechanics (FEM/FEA), Composite Materials, Programming \\
		Tools & Python, Siemens NX (CAD), Simcenter 3D (CAE), Nastran, Samcef, Git \\
		Languages & Dutch (Native), English (Fluent), French (Basic) \\
	\end{tabular}

\end{rSection}

%----------------------------------------------------------------------------------------
%	INTERESTS SECTION
%----------------------------------------------------------------------------------------

\begin{rSection}{Interests}
	Dog walking, playing guitar, reading (non)-fiction, board games, space(flight), building a pumkin chunkin launcher
\end{rSection}

%----------------------------------------------------------------------------------------

\end{document}
